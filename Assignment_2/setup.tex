\section{Infrastructure Setup}
We identify four main services which require machine-to-machine communication and, thus, are susceptible to the firewall rules that we'll define:\\
\begin{itemize}
\item an \textbf{Apache} web server reachable through \textbf{TCP} ports \textbf{80} (HTTP) and \textbf{443} (HTTPS) at \textbf{100.100.6.2};
\item a \textbf{DNS service} reachable through standard \textbf{UDP} port \textbf{53} at \textbf{100.100.1.2};
\item a \textbf{syslog server} reachable through standard \textbf{UDP} port \textbf{514} at \textbf{100.100.1.3};
\item a \textbf{proxy server} reachable through port \textbf{3128} at \textbf{100.100.6.3}, to be configured later on Assignemnt 3.
\end{itemize}

We thus identify two new subnetworks: the \textbf{Internal Servers} subnetwork \textbf{100.100.1.0$/$24} and the \textbf{DMZ} subnetwork \textbf{100.100.6.0$/$24}, and for the sake of defining the scope of this assignment and the setup of the whole infrastructure, we should also consider two additional details:\\
\begin{itemize}
\item Two of the four services - web server and syslog server - are already setup and running in the system: they do not need to be configured or launched;
\item the \textbf{proxy server} is to be configured on the next assignment: for now, knowing from the \textbf{zentyal} portal that it is provided on port 3128 is enough.
\end{itemize}

This leaves us with the only \textbf{DNS service} to be configured at the moment. This step has been performed by following the instructions provided with the assignment: by accessing the \textbf{zentyal} portal at \textbf{100.100.1.2} on port \textbf{8843} - credentials have been changed - we added as forwarders the suggested IP addresses and specified the domain name - \textbf{acme.group27}. Then, the file located at \textbf{$/$etc$/$zentyal$/$dns.conf} on the aforementioned machine was modified including the target subnetworks (Clients and DMZ, that are the ones which will exploit the service).\\
At this point, pairs of IP addresses and hostnames were specified in the portal, so that the well-known machines of the network can now be associated with the following names:\\
\begin{itemize}
\item \textbf{kali}.acme.group27;
\item \textbf{watchdog}.acme.group27;
\item \textbf{dc}.acme.group27;
\item \textbf{web}.acme.group27;
\item \textbf{proxy}.acme.group27;
\item \textbf{coffee}.acme.group27;
\end{itemize}

also, as suggested in the assignment, the external services machines were provided with an external DNS service such as 8.8.8.8 in their DHCPv4 configuration. Please notice also that the \textit{logserver} was not given a name since it is only meant to be accessed by the SPOCK environment or by SSH and is not offering any browser-related service - and the same reasoning could be applied to the first two machines, \textit{kali} and \textit{arpwatch}, which we could exclude from this pairing list.\\
Last step was to actually modify the \tetxbf{DHCPv4} settings in the two main routers to set the \textbf{dc} machine as the \textbf{DNS server} in the Internal Servers network, after having disabled the \textbf{Service Unbound DNS} in both routers.\\
The \textbf{DNS service} configuration is tested in section 5.
