\section{Policy Implementation}
While evaluating the policy in the previous paragraph, we have already proposed a high-level description of the rules to be implemented at each interface of the two firewalls. Notice that, for isntance, the rules we applied at the \textbf{Internal interface} of the \textbf{Main Router}, could have also been applied, instead, at the \textbf{External interface} of the \textbf{Internal Router} or, to have a complete in-depth defence, the same rules could have been applied at both interfaces - there is an option, indeed, to group rules applied to multiple interfaces in OPNSense - which could be the best solution in order to prevent the firewall to be taken down when one of the two machines is not working properly.\\
Also, there is another interesting fact about OPNSense firewall configuration: it tracks connections by default, so that every packet in a TCP connection which has already been \textit{established} and has been accepted, is automatically accepted by default. This means that if we set a rule to allow machine A to ping machine B through ICMP protocol, machine B's response will be automatically accepted too by the firewall, while if machine B initiates a new connection by pinging machine A, the firewall will not allow it - and we can verify this on the testing paragraph.\\
