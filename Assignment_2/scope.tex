\section{Scope and Initial Considerations}
The scope of this assignment comprehends the whole target network: a series of services provided by different machines in several subnetworks in the \textbf{ACME} environment are listed, and we focus on how to enable these services to the clients of our network by exposing them on a controlled manner through firewall rules specified in the \textbf{OPNSense} service provided by the two main routers of the network.\\
By catching a quick glimpse of the policy proposed by the assignemnt, we can tell the scope is to configure our network with a series of \textit{white-listed} services that we want to be enabled and reachable from certain machines in the network itself, thus reducing the \textit{attack surface} of the network by also excluding from this white list every other service that is not of interest.\\
Our analysis should begin with the services that we want to provide in the target network, and then on the firewall rules that make it possible to reach each service on each different machine.\\
Please notice that, since IPv6 addresses are not part of the scope of this assignment, we will ignore them during the configuration of the firewall rules in section 4.\\
