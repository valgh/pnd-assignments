\section{Final remarks}
The testing phase showed us that the policy, if correctly interpreted, was implemented as specified by the assignment.\\
Some extra measures could have been taken, and they are very similar to the ones we have seen in the first assignment: we need to perform Linux hardening via sudo and hardening of the SSH protocol for all the machines in the DMZ and Internal Service subnetworks: this was done by following the same procedure that was applied to the \textbf{arpwatch machine} in the previous assignment, thus by disabling SSH login via basic authentication on the \textit{root} account of every machine, and by enabling only \textit{public key authentication} for this very same protocol through the \textit{sshd$\_$config} file. In the machines that do not exploit \textbf{Zentyal} thus, another user was created and managed via \textit{sudoers} file - on the \textit{logserver} and \textit{webserver} machines. The machines are thus only reachable by the \textbf{Kali machine} - the only machine holding the authorized key, protected by a passphrase - through SSH login.\\
We should also stress the fact that the scope of this assignment was to focus mainly on the \textbf{firewall rules} to be defined to achieve our goals and implement the given policy: this means that link$-$layer attacks were not taken into consideration, and this may be a critical vulnerability in our network. indeed, notice that every machine can be accessed through SSH or other protocols by any other machine on its same subnetwork - i.e., \textit{dc} can access \textit{logserver} through any protocol. This is because the rules we defined to implement our policy are placed on the two firewall$-$routers, which obviously do not handle packets exchanged between hosts in the same subnetwork: to overcome this issue, probably some extra rules should also be defined locally on each machine via \textit{iptables} - but this was out of our scope for this assignment.\\
We should also mention the fact that the \textbf{UDP} protocol was chosen to implement the \textbf{rsyslog} service, which could be also implemented through \textbf{TCP} protocol - commonly considered more secure - and maybe forced to exploit secure channels. Again, our goal here was to enable the service and enforce the policy via firewall rules, but we should stress the fact that just by exploiting TCP, this service may be hardened.\\
Also, as a last step, when the whole configuration going on in these assignments is over, we should at least enable only one machine in the \textbf{Clients} subnetwork - the \textbf{Kali} machine, probably - to connect via HTTP service to the two routers, for administration and practical reasons, so as we mentioned in the previous paragraphs, the rule enabling these connections should be actually disabled as a last step.
