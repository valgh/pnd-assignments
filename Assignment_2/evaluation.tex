\section{Policy Evaluation}
The proposed policy targets the four services listed in the previous paragraph and a series of machines which will either exploit or provide the corresponding services.\\
The best way of interpreting and understanding this policy is given by its fifth line:\\
\begin{itemize}
\item \textit{"Anything that is not specifically allowed has to be denied"};
\end{itemize}

which suggests the \textit{white-listed} approach we have to adopt when defining the firewall rules: the policy is a list of \textbf{PASS} rules - meaning, rules that when matched will let the packet pass and continue its journey - while everything that doesn't match the rule has to be rejected, i.e. the packet must be dropped.\\
If implemented correctly, this policy should allow the Internal services to be exploited and reachable only by the \textbf{Clients or DMZ networks}, and it should also allow the internal client hosts in the \textbf{Clients} network to either reach the \textbf{external web services} through HTTP$/$HTTPS, or the Internet via the \textbf{Proxy service} to be configured in the next assignment on port 3128, being this service in the \textbf{DMZ} the only one that is able to actually initiate connections with \textbf{WAN} again on HTTP$/$HTTPS protocols, while the \textbf{web server} in \textbf{DMZ} should be the only machine which can accept connections initiated by someone else from the Internet and thus outside our target network.\\
Also, every machine in the internal network (meaning \textbf{DMZ}, \textbf{Internal servers} and \textbf{Clients network}) should be accessible on port 22 (SSH) only from hosts in the \textbf{Clients network} and obviously reject any other connection on that port from undesired hosts: this means only the internal clients in \textbf{100.100.2.0$/$24} should be actually able to administrate the machines in our internal network.\\
Furthermore, external clients may freely choose their DNS service, so they can also request an external DNS service on the Internet, but every other machine in our internal network must exploit our internal DNS Service - and, even if this can be bypassed easily, this means that internal machines are only able to reach by their domain names machines whose names are defined in our internal DNS service, at least at the very beginning.\\
The two main routers, other than having their default passwords changed, should also only be reachable on their HTTP service by machines in the internal clients network.\\
Notice that there may be several different working setups to actually implement the policy, and the one we implement and describe in the next paragraph may not be the only one.\\
Nonetheless, we can generally state that the approach that is suggested, given the infrastructure's architecture, is that of having a \textit{dual-homed host} firewall approach - indeed, we do have two firewalls, and the \textbf{Internal} interface of the \textbf{Main Firewall} especially should decide whether packets coming from the \textbf{WAN} should actually be received by the \textbf{Internal Firewall} and viceversa - obviously with a \textbf{DMZ} placed in between to implement a \textbf{defense-in-depth} strategy.
