\section{Policy Evaluation}
The proposed policy targets the four services listed in the previous paragraph and a series of machines which will either exploit or provide the corresponding services.\\
The best way of interpreting and understanding this policy is given by its fifth line:\\
\begin{itemize}
\item \textit{"Anything that is not specifically allowed has to be denied"};
\end{itemize}

which suggests the \textit{white-listed} approach we have to adopt when defining the firewall rules: the policy is a list of \textbf{PASS} rules - meaning, rules that when matched will let the packet pass and continue its journey - while everything that doesn't match the rule has to be rejected, i.e. the packet must be dropped. Thus, we can devise a list of rules to apply at each of the two target routers and their corresponding effects on the network.\\

\textbf{Internal Router:}
\begin{itemize}
\item \textbf{Clients Interface}: only accept incoming packets on this interface with destination ports 80(HTTP), 443(HTTPS), 22(SSH), 53(UDP-DNS) and 3128(Proxy Service). Anything else will be discarder, since clients are not supposed to perform different actions and exploit different protocols than the aforementioned ones. For practical reasons, also packets on port 8443 (for \textbf{zentyal} panel service) are allowed - but this might be changed;
\item \textbf{Servers Interface}: only accept incoming packets on this interface with destination port 53(UDP-DNS) or 514(UDP-SYSLOG, only if coming from \textbf{DMZ subnet}) and on port 22(SSH) only if coming from the \textbf{Clients subnet}, otherwise discard. This means that the two servers can only be reached by Clients or DMZ subnetworks, and only for the services that they provide (SYSLOG, DNS, SSH);
\end{itemize}

\textbf{Main Router:}
\begin{itemize}
\item \textbf{Internal Interface:} only accept incoming packets on this interface on ports 80(HTTP) and 443(HTTPS) between \textbf{Clients subnet} and \textbf{External services subnet}, or ports 22(SSH) and 3182(Proxy) between \textbf{Clients subnet} and \textbf{DMZ subnet}, or ports 53(UDP-DNS) and 514(UDP-SYSLOG) between \textbf{DMZ subnet} and \textbf{Internal services subnet}, otherwise discard. This means the Client host will only be able to reach the external web services, or the proxy to actually reach the WAN, and won't be able to do it directly by itself. Furthermore, SSH is eanbled for Client hosts to reach the \textbf{DMZ subnet} and the services offered by the \textbf{Internal servers} are reachable from the machines in the DMZ;
\item \textbf{DMZ Interface:} we want the proxy server to be able to reach the internet via HTTP$/$HTTPS protocol, so we specify that this interface has to accept incoming packets specifically from the \textbf{Proxy server} and with \textbf{any} destination address on ports HTTP$/$HTTPS, and the same goes (but reversed) for the \textbf{web server} which we want to be accessible on ports HTTP$/$HTTPS from the Internet. However, the \textbf{proxy service} itself must be only available for client hosts in the \textbf{Clients subnet}, so we should also specify that this interface should accept incoming packets on port 3182 of the Proxy machine with source address in the \textbf{Clients subnet}. The aforementioned rules for DNS/SYSLOG services in the Internal Services apply also here on this interface, while every other packet which is not \textit{white-listed} must be rejected;
\item \textbf{WAN Interface:} this interface should be the first line of prevention against intruders from the Internet, so it only has to accept - as specified by the policy - connections with destination port 80 or 443 on the \textbf{web server} machine in DMZ and connections to any other machine in the Internet on port 80 or 443 if they are initiated by the \textbf{Proxy} machine in DMZ: every other packet incoming on this interface should be rejected.
\end{itemize}

If implemented correctly, this policy should allow the Internal services to be exploited and reachable only by the \textbf{Clients or DMZ networks}, and it should also allow the client hosts in the \textbf{Clients} network to either reach the \textbf{external web services} or the Internet via the \textbf{Proxy service} to be configured in the next assignment, being this service in the \textbf{DMZ} the only one able to actually initiate connections with Internet, while the \textbf{web server} should be the only machine which can accept connections initiated by someone else outside our target network.\\
Notice that this may not be the only possible setup to actually implement the policy, and we will deal with this in the next paragraph.
