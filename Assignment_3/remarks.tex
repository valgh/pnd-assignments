\section{Final remarks}
\textbf{OpenVPN} and \textbf{Squid Proxy} services were both configured and tested successfully during this assignment, and we believe their deployment comply with the policies and requirements provided by the assignment itself.\\
One point that we want to stress again though, is the fact that both services rely (\textbf{squid} especially, the \textbf{VPN} seems more protected since it also implements \textit{TLS auth} and a \textit{OTP}) on some form of \textbf{basic authentication}, which is commonly known to be the weakest form of authentication. That's why these two services should be supported by other protection layers, such as proper \textbf{firewall rules} for both \textbf{WAN interface} and \textbf{Proxy machine}, and \textbf{Public Key based Authentication} at least for every \textbf{SSH service} in the internal subnetworks of our target network. Those passwords are, at least, stored in an encrypted manner as described in the previous paragraph about proxy authentication.
Some of the measures that are taken in the next and last assignment are, indeed, also meant to protect these two services - e.g. \textit{fail2ban} for \textbf{squid}.\\
