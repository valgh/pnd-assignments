\section{Authentication}
Both services we implemented in this assignment require a \textit{basic authentication} in order to be accessed. Users to be authenticated are the three we already mentioned in the previous paragraphs, belonging to the \textbf{road warriors} set: \textit{Becca, Huck and Jim}. Each of them has two unique passwords, different from each other, one for the \textbf{VPN service} and one for the \textbf{Proxy service}.\\

\subsection{VPN Authentication}
Access rules for the \textbf{VPN service} have been defined on the \textbf{OPNSense} administration panel located at the \textbf{Main router}, under the \textbf{System} tab. The three users have been assigned to the \textbf{ROADWARRIORS} group and, for each of them, an \textbf{RSA Public Key} has been created and assigned for \textbf{TLS Authentication} process of \textbf{OpenVPN}. Moreover, each of the three users holds a unique password and a \textbf{One Time Password} - a six-characters token provided by \textbf{Google Authenticator}.\\
This way, in order for one of the users to authenticate to the \textbf{OpenVPN service}, it must necessarily own \textit{three requirements}:\\
\begin{itemize}
\item its own unique \textbf{.ovpn client exported file} located under the tab \textbf{OpenVPN $->$ Clien Export}, which must be exploited by the client machine's \textbf{openvpn} tool to retrieve the server's location, port, parameters and perform the \textbf{TLS Authentication};
\item its own unique and static \textbf{password} defined under its own profile on the \textbf{OPNSense} administration panel;
\item its own \textbf{OTP} provided by \textbf{Google Authenticator}
\end{itemize}

when these three requirements are met, the user can easily access to the \textbf{OpenVPN} service and reach the internal machines as desired.

\subsection{Proxy Authentication}
As already mentioned, the \textbf{Proxy Service} is only reachable when the client's IP address is found to belong to the \textbf{Clients subnetwork} (\textit{100.100.2.0$/$24}).\\
Moreover, we perform once again \textbf{basic authentication} so that only internal clients owning a \textit{username-password} pair that is known to the service can actually authenticate and exploit it. Once again, the three authorized users to authenticate to the service are \textit{Becca, Huck and Jim}. Each of them holds a unique password, different from the ones we defined for the \textbf{OpenVPN service}, whose \textit{hash} has been stored through the \textbf{htpasswd} tool under the file \textit{$/$etc$/$squid$/$passwd} located at \textbf{proxy.acme.group27}, so that the passwords are not stored in clear.\\
This way, each time an user from the \textit{internal Clients subnetwork} accesses a web browser having the \textbf{Proxy service} set, a dialog pops up requesting username and password to authenticate: if the authentication fails, the service won't be reached and the user will not be able to exploit it and thus surf the Internet.\\
Details about the basic authentication exploited in the \textbf{Squid} service were also given in the previous paragraph: passwords are \textit{case sensitive} and the service only remembers an authenticated user for \textit{2 hours} if he doesn't disconnect from the service itself. After 2 hours, the service will require the user once again to provide their credentials.
