\section{Final remarks and possible improvements}
Some improvements are still possible to this setup: one issue that we may not be able to actually address is the possible physical shut down of the \textbf{arpwatch-clients} machine, which may lead to the monitoring task not being carried out.\\
Some extra-measures were also applied to disable SSH login on the \textit{root account} on the arpwatch machine, so that it is only possible to login on the \textit{watchdog account} via \textit{public key} authentication - \textit{basic authentication} has been disabled - and the only public key that is accepted is owned by the \textbf{Kali machine} and protected by a passphrase. The changes to the \textit{ssh daemon} were made through the \textit{sshd$\_$config} file under the \textit{$/$etc$/$ssh$/$} directory.\\
Also, the ether packet that is sent by the script we built is not properly secured, i.e. it is not sent through an encrypted channel, meaning that everyone listening inside the subnet may be able to access its content. This, and the lack of an appropriate check on the algorithms exploited by the SSH protocol, are to our knowledge the only best practices that were not performed in the hardening process of the hosts in the subnetwork, since all the default passwords on the hosts were also either changed or disabled.
Nonetheless, some of the basic security measures were implemented to enforce security on the subnetwork's machines, and our main goal - to monitor the target LAN through \textbf{arpwatch} - has been carried out successfully.
