\section{Final remarks and possible improvements}
Some improvements are still possible to this setup: one issue that we may not be able to actually address is the possible physical shut down of the \textbf{arpwatch-clients} machine, which may lead to the monitoring task not being carried out.\\
Some extra-measures could be taken to disable SSH on the arpwatch machine or, at least, to enable it only via \textit{public key} authentication and disabling the \textit{basic authentication} via password, even if these were changed from their default ones. To keep using \textit{basic authentication} on SSH, a feasible solution would be to implement a password policy that forces users on the \textbf{arpwatch-clients} machine to change their password frequently - say, at least every week.\\
Another improvement that should be performed, is that of disabling the \textbf{root} account on every machine in the subnetwork - \textbf{root} is a well-known, default super user that almost every Unix machine has, so it's a well-known target for attackers.\\
Also, the ether packet that is sent by the script we built is not properly secured, i.e. it is not sent through an encrypted channel (no TLS), meaning that everyone listening to the network is able to access its content, but this may be a minor issue.
Nonetheless, some of the basic security measures were implemented to enforce security on the subnetwork's machines, and our main goal - to monitor the target LAN through \textbf{arpwatch} - has been carried out successfully.
