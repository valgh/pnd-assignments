\section{Final remarks and possible improvements}
Some improvements are still possible to this setup: one issue that we may not be able to actually address is the possible physical shut down of the \textbf{arpwatch-clients} machine, which may lead to the monitoring task not being carried out.\\
Some extra-measures were also applied to disable SSH login on the \textit{root account} on the arpwatch machine, so that it is only possible to login on the \textit{watchdog account} via \textit{public key} authentication - \textit{basic authentication} has been disabled - and the only public key that is accepted is owned by the \textbf{Kali machine} and protected by a passphrase. The changes to the \textit{ssh daemon} were made through the \textit{sshd$\_$config} file under the \textit{$/$etc$/$ssh$/$} directory.\\
Another improvement that should be performed, is that of disabling the physical access to the \textbf{root} account on every machine in the subnetwork - \textbf{root} is a well-known, default super user that almost every Unix machine has, so it's a well-known target for attackers.\\
Also, the ether packet that is sent by the script we built is not properly secured, i.e. it is not sent through an encrypted channel (no TLS), meaning that everyone listening to the network is able to access its content, but this may be a minor issue.
Nonetheless, some of the basic security measures were implemented to enforce security on the subnetwork's machines, and our main goal - to monitor the target LAN through \textbf{arpwatch} - has been carried out successfully.
