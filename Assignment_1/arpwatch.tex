\section{Arpwatch tool configuration}
This section is about how the \textbf{Arpwatch} tool was configured in the corresponding machine.\\
It is composed of two subparagraphs: the first one briefly describes the tool's configuration, while the
scripts that were written in order to monitor the target network are described in the second one. Note that the security issues that we had to face and the countermeasures taken for this configuration to be actually effective are described in the next section as part of additional measures that were not required by the assignment, but somehow needed to secure the whole \textbf{Clients network}.\\

\subsection{Arpwatch configuration}
The \textbf{Arpwatch} tool was not installed by the default in the corresponding machine, so it had to be downloaded and installed via command \textit{apt install arpwatch}. The empty file \textbf{eth0.iface} had also to be created in the tool's directory \textit{/etc/arpwatch/} in order for \textbf{Arpwatch} to be able to listen through the machine's interface linked to the \textbf{Clients network}'s LAN.\\
With the default configuration, the tool is already able to detect possible changes in the network when launched, and mail them to the \textbf{root} - and this function is still enabled at the end of the whole configuration, mails are stored under \textit{/mail/root/}.\\
In order for the configuration to be effective, it is enough to launch the tool in our scripts via command \textit{arpwatch -i eth0 -n 100.100.2.0/24} so that it will listen to the changes in the \textit{whole} network, meaning if any machine in the \textbf{Clients network} pings any other machine (not just the arpwatch one) in the aforementioned network, the tool will be able to eventually capture potential spoofing attempts and send them an alert.\\
\textbf{Arpwatch} usually logs any captured event in a Unix predefined log file under \textit{/var/log/messages}, but this may be inconvenient for us since this file holds every single log that was generated by the machine, and we are only interested in monitoring \textbf{Arpwatch}-generated logs. This is the reason why we chose to define a new file where the tool stores its own logs, by configuring the \textbf{rsyslog} tool installed by the default to actually redirect all the logs generated by \textbf{Arpwatch} in their own file, which can be found under \textit{/var/log/arpwatch.log}.\\
The aforementioned configuration can be found under \textit{/etc/rsyslog.d/arpwatch.conf}.\\

\subsection{Monitoring scripts}
Three scripts were written to actually monitor our target LAN: the first one, \textit{start$_$monitoring.sh} is located under \textit{/etc/profile.d/} and it is only needed to launch at login  the second one, located at \textit{/home/bash$_$shell$_$script.sh}. This script is actually the core of the whole monitoring system: it launches \textbf{Arpwatch} as previously described, then it tails its log file so that, whenever a new line is generated, if it is \textit{interesting} - meaning it possibly detects an intrusion event in the monitored network, thus reporting a \textbf{MAC address} - it sends as input to the third script the detected \textbf{MAC address} and a \textbf{payload} that is, as required by the assignment, the whole log's new line.\\
The third script is a \textbf{Python script} named \textit{scapySendPacket.py} which exploits \textbf{ScaPy} module to generate a new ethernet packet to be sent to the detected address thanks to the input received. We have to highlight at least two items at this point:\\

\begin{itemize}
\item the packet has to be sent with a header \textbf{type} field that is \textit{different} from \textbf{0101}, the one requested by the assignemnt: this is due to the fact that the implementation of the \textbf{ScaPy} module only allows to specify values for this field that are \textbf{>= 0x0600}, otherwise they are interpreted as values for the \textbf{length} field. To overcome this problem, we decided to assign to this field the value \textbf{0x88b5}, defined by the \textbf{IEEE} as \textbf{Local Experimental Ethertype} - so still an experimental type, at least;
\item this \textbf{Python} script is tghe only one that takes some values as input; furthermore, it is the only one actually performing network operations in our configuration, so we have to keep this in mind since it may be the only script that actually requires some kind of input validation in order to be considered as secure.
\end{itemize}

The execution of the last script obviously required to install \textbf{python3}, \textbf{pip3} and \textbf{scapy} module via \textbf{pip3} on the arpwatch machine.
