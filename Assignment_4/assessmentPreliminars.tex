\section{Engaging the Assessment}
In this section we are going to define the \textbf{scope} of the assessment, as weel as the typology of assessment we are going to carry out - i.e., which subnetworks we are going to test, and how prepared these subnetworks should be. Since we want to base our test upon the voices we heard in the company, if there is a new platform that is being developed, it is very likely to be placed in the \textbf{DMZ subnet}: indeed, it is likely that a platform belonging to our company is going to accept connections from visitors or external partners, and the \textbf{DMZ} is where this should happen.\\
Furthermore, our platform is surely going to exploit the services provided by our \textbf{Internal servers} - i.e. the logging service and the DNS service. This is why we should focus our assessment on these two subnetworks, so \textbf{DMZ} and \textbf{Internal servers} are going to be the subnetworks which define the perimeter of our assessment, our scope.\\
While the operations of \textbf{Linux hardening} we performed previously should not be rolled back during this assessment, we should ask ourselves how the other security measures we implemented are going to affect the tests als with respect of the typology of assessment we want to make. In order to gather as much information as possible about these two subnetworks, indeed, we should carry out the assessment with almost every security measure in the targets disabled. This is also due to the fact that the \textbf{GSM} machine is actually inside one of our target subnetworks (\textbf{Internal servers}), and while for the hosts in this subnet security measures such as firewalls or IPS are not going to affect the scan - traffic between scanning machine and targets is not going through any firewall at all - these measures could actually affect the scanning procedure of hosts in the \textbf{DMZ} subnet: indeed, we tried and found out that, with the firewall rules enabled, \textbf{GSM} is not able to perform the scan. This means that, to actually be able to find vulnerabilities in that subnetwork, we should at least get rid of the firewall - as if the \textbf{GSM} machine was in the same subnet. Obviously it is not, so the vulnerabilities that we are going to find out, have to be considered such in the worst case scenario - that is, an adversary who gained access to these two subnetworks and can send/receive Ethernet packets in there, for example gaining phyisical access to them or by performing some kind of spoofing attack.\\
The assessment results that we are going to describe and evaluate in the next section are, thus, obtained on the target hosts \textit{with firewalls and IPS (fail2ban and suricata) disabled}.
